\documentclass[a4paper]{article}
\usepackage{amssymb}
\usepackage{amsmath}
\usepackage{graphicx}

\graphicspath{{./figs/}}

\begin{document}

\section*{(A)}
The autocorrelation for sequance of elements is given by: $cor(x_t, B^k (x_t))$ 
and that the back shift operator is defiend as: $ B^{k}(x_t) = x_{t-k} \text{ and } B^{-k}(x_t) = x_{t+k} $
\begin{align*}
    acf(k) &= cor(x_t, B^k (x_t)), t > k \\
    &= cor(x_{x_{t + k}}, B^k(x_{t + k})) \\
    &= cor( B^{-k}(x_{t}), B^{-k}(B^k( x_t )) ) \\
    &= cor(B^{-k}(x_t), x_t) \\
    &= \tilde{acf}(k)
\end{align*}


\section*{(B)}
By following the same logic from A for acf, then ccf fail and based on 
that result we conclude that $\tilde{ccf}(k) \neq ccf(k)$, and is showed her:
\begin{align*}
    \tilde{ccf}(k) &= cor(x_t, B^{-k}(y_t)) \\
    ccf(k) &= cor(x_{x_{t}}, B^k(y_{t})) \\
    &= cor(x_{x_{t + k}}, B^k(y_{t + k})) \\
    &= cor( B^{-k}(x_{t}), B^{-k}(B^k( y_t )) ) \\
    &= cor(B^{-k}(x_t), y_t) \\
    & \neq  cor(x_t, B^{-k}(y_t)) = \tilde{ccf}(k)\\
\end{align*}
The back shift operator has changed feature to shift and diraction, which result 
in the opposite effect that we are looking for $\tilde{ccf}$. When we use the method from A 
we go from backshofting y to forward shift x, when $\tilde{ccf}$ forward shift y witch is 
the opposite direction and reflect the $ccf(k)$

\section*{(C)}
\begin{itemize}
    \item (A): Time series A have ACF 3 as its auto correlation plot. Based on the that 
    TS1 is only noise and the only point of correlation are for 0 lag, and every other lag 
    has random and insignificant correlation.
    \item (B): Time series B have ACF2 as its ait correaltion plot. Based on linear drop of corelation for 
    every increments of lag.
    \item (C): Time Series C have ACF as its auto correaltion plot. Based on the shape 
    of TS 3 can be represented as a cosine fucntion with some noise. witch will result in 
    a smooth changh in correaltion for each lag and negavtiv corealtion when the the distibution is 
    lag with the same lenght of half a periode
    \item (D): Time series D hace ACF 1 as its auto correaltion plot. Based on the fact that 
    TS 4 can be reprecentet by two different cosine function. one with long periodes and on short. This wil 
    result in a big and smale fluctuation on the bigger curve of the auto correaltion.
\end{itemize}

\section*{(D)}
The Autocorrelation of TS3 and TS5 wil be the same, because TS 5 is lagged version of 
TS 3. if we set TS3 as x feature and TS5 as y feautre. For each backshift of y will make the time series more similar.
By the visualizations it seems to me lag with about 15 increments. So the cross corelation plot will  start with a correaltion around 0. 
and will for each lag increas the correlation until both time series are synchronised for so decrease for each lag. 
The shape wil then look like for most of the plot like a half circle.

\end{document}
